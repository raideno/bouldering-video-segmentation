\newpage
\section{Context}
\label{section:context}

\noindent\textbf{Bouldering Basics.} 
It is a form of rock climbing where climbers tackle short but challenging routes, typically lasting around 4 minutes. During this time, climbers have the freedom to choose their climbing strategies, retry as many times as needed, and aim to complete the route as quickly as possible. The ultimate goal is to reach the top of the climb in the least amount of time.

\noindent\textbf{Project Overview.} 
This research project is part of a larger initiative (ANR) aimed at improving bouldering performance. The focus is on developing tools for coaches to analyze bouldering performances, allowing them to provide better advice to help climbers improve.

\noindent\textbf{Existing Tools.} 
Among these tools are: Grip Detection, Path Tracking, Performance Analysis, and more. Several models have already been developed as part of this broader effort, including the model for detecting the grips used by the climber and the climber's climbing path tracking model.

\noindent\textbf{Phase Identification.} 
These models provide valuable data, but in order to fully analyze a climber's performance, it is essential to distinguish between the different phases of a bouldering event (Climbing, Observing, Brushing). By doing so, we can apply the appropriate model to each phase and gather the corresponding statistics.

\noindent\textbf{Research Focus.} 
This brings us to the current research project. The project will focus on Temporal Action Segmentation (TAS), a technique that will enable us to identify the different phases of a bouldering event. By accurately distinguishing between phases such as observing, reading the route, and brushing the grips, we can gain deeper insights into the climber's performance. Analyzing the duration, order, and impact of these phases will allow coaches to provide more targeted guidance and improve training methods.

\begin{figure*}[!t]
    \centering
    \begin{tabular}{@{}c@{\hspace{15pt}}c@{\hspace{15pt}}c@{\hspace{15pt}}c@{}}
      \setlength{\fboxsep}{0pt}
      \fbox{\includegraphics[width=0.2\textwidth]{../../assets/images/observing.2.png}} &
      \setlength{\fboxsep}{0pt}
      \fbox{\includegraphics[width=0.2\textwidth]{../../assets/images/climbing.1.png}} &
      \setlength{\fboxsep}{0pt}
      \fbox{\includegraphics[width=0.2\textwidth]{../../assets/images/brushing.3.png}} &
      \setlength{\fboxsep}{0pt}
      \fbox{\includegraphics[width=0.2\textwidth]{../../assets/images/climbing.2.png}} \\[6pt]
      (a) Observing the block. &
      (b) Climbing the block. &
      (c) Brushing the grips. &
      (d) Climbing the block.
    \end{tabular}
    \caption{Different phases of bouldering.}
    \label{figure:phases-of-bouldering}
  \end{figure*}