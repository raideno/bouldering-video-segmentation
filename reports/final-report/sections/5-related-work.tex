\section{Related Work}

This will be a sort of state of the art.

We do an overview of TAS methods and other formulations methods.

We talk about the limitations in the current methods (not enough data and compute power available, not enough diverse data, as seen most of the datasets are youtube videos).

Illustration that describe the evolution of the different methods with the evolution of dataset size and compute power.
Hand Features - Optical Flow - 2D CNNs - 3D CNNs
Talk about the different approaches and techniques. Talk about their evolution just like it is done in \textbf{ViViT: A Video Vision Transformer} paper. (First started by using hand collected features, 2D CNNs, 3D with optical flow, full 3d, transformers as the datasets started to grow, etc.)

\todo[inline]{We'll need an illustration here.}

\todo[inline]{We could also have a cahrt showing the evolution of accuracy and number of papers on the field with time, and on the time axis we put markers for different events such as the release of popular datasets, improvements in compute power, etc.}

\todo[inline]{We also need to talk about the old school methods, how they used optical flow, etc}

\todo[inline]{Maybe this should be moved elswhere, i feel it is not it's place and should be presented before, maybe in the formulations section}
Talk about the importance of TAS and its applications.

Talk about the limitations of this presented methods, their constraints in terms of dataset size and required compute power.

NOTE: the models tend to drift much faster when it comes to videos data as the distributions are much much larger and thus it requires even more data  probabily.

NOTE: recall that it is a hard problem and that video data is very very varied and diverse and that distributions aren't the same and that even pre trained models don't work very well on other data then thaier datasets, and it is an on going research area.