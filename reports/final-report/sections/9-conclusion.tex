\newpage
\section{Conclusion}

\subsection*{Research Contributions}
This work demonstrates the feasibility of automated bouldering video segmentation using modern deep learning approaches. Our experiments revealed that segment-based models, particularly the X3D family and the R3D, outperform frame-based approaches for this task, achieving accuracies of up to 86.80\%. We established that the choice of pre-training dataset significantly impacts performance, as evidenced by the stark difference between S3D models trained on HowTo100M versus Kinetics. Notably, model complexity did not necessarily correlate with performance, with lighter models often matching or exceeding their larger counterparts.

Future research should focus on improving real-time capabilities, investigating alternative loss functions, and exploring transformer-based architectures for video understanding. Semi-supervised approaches leveraging unlabeled climbing footage could address the data scarcity challenge while maintaining evaluation integrity.

\subsection*{Project Reflection}
This project provided valuable experience in end-to-end data science development. Building a computer vision system from scratch presented numerous challenges, from data collection and annotation to model selection and evaluation. The need to balance performance requirements against computational constraints mirrors real-world AI deployment scenarios.

The most significant challenge was navigating the vast landscape of video understanding models without extensive training resources. This required careful consideration of transfer learning approaches and creative solutions for data augmentation.

This work served as a formative experience in independent research, requiring comprehensive exploration of the computer vision literature and adaptation of techniques across domains. It demanded proficiency across the entire machine learning pipeline—from data preparation to deployment considerations—providing practical skills beyond what typical structured projects offer. This holistic approach to solving a complex visual understanding problem has reinforced both technical capabilities and research methodology fundamentals.