\section{Problem Formulations, Scores \& Evaluation Metrics}

In this project, we address the task of temporal action segmentation in professional bouldering videos. The objective is to classify each frame of a video based on the climber's activity.

\subsection{Video Representation}
We represent each video as a sequence of frames $video_i = \{f_1, f_2, \cdots, f_N\}$ where $f_i$ is a frame represented by a tensor of size $C \times H \times W$, with $C$ being the number of channels (fixed to 3 for RGB videos), $H$ the video height, $W$ the video width, and $N$ the total number of frames in the video.

The video's duration (in seconds) can be calculated as $\text{Duration} = \frac{N}{F}$ where $F$ is the video's frame rate which will be fixed to $25$ during our experiments.

\subsection{Annotation Representation}
Each video's annotations are provided at the frame level. Given a video $video_i$, its corresponding annotations are defined as $a_i = \{c_1, c_2, \cdots, c_N\}$ where $c_i \in C$, and $C$ is the set of all possible activity labels. In our case $C = \{\text{Observing}, \: \text{Brushing}, \: \text{Cleaning}, \: \text{Stopwatch} \}$

\subsection{Problem Formulation}
The goal is to develop a model $f$ that takes a video as input and predicts the corresponding sequence of frame-level annotations $f : video_i \rightarrow a_i$. Alternatively, the model can be designed to:

\quad \emph{1. Frame-wise Prediction:} The model processes each frame independently and predicts its corresponding label:  
    $f_{frame} : f_t \rightarrow c_t$.

\quad \emph{2. Sequence-based Prediction:} The model processes a sequence of $T$ frames and predicts a single label for the entire sequence:  
    $f_{seq} : \{f_{t}, f_{t+1}, \cdots, f_{t+T-1} \} \rightarrow c_t$.

The choice between these formulations may depend on the desired balance between temporal context awareness and computational efficiency.

\todo[inline]{Read papers and check for other possible alternative formulations.}

\subsection{Statistics}

Scores and statistics to better explore the data and videos.

For the data we'll be mainly using two statistics to access information about our data, these two statistics are \textbf{Repetition Score} and \textbf{Order Variation Score} that have been introduced in \cite{tas-survey}.

\textbf{Repetition Score}: Definition..

\textbf{Order Variation Score}: Definition..

\subsection{Metrics \& Scores}

In order to access the model's performance, the most commonly used metrics in the literature are the following:

- MoF (Accuracy): Define.., Pros, Cos.

- IoU: Define.., Pros, Cos.

- F1: Define.., Pros, Cos.

- Precision: Define.., Pros, Cos.

- Recall: Define.., Pros, Cos.