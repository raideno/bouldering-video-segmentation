\documentclass{article}

\usepackage{microtype}
\usepackage{graphicx}
\usepackage{subfigure}
\usepackage{booktabs} % NOTE: for professional tables
\usepackage{makecell}
\usepackage{listings}
\usepackage{xcolor}
\lstset{
  breaklines=true,              % NOTE: enable automatic line breaking
  basicstyle=\small\ttfamily,   % NOTE: adjust the font size and type
  escapeinside={||},            % NOTE: define escape characters for LaTeX commands
}
\usepackage{fancyvrb}
\usepackage{minted}
% Set options for minted:
\setminted{
  breaklines=true,   % Enable automatic line breaking
  fontsize=\small,   % Adjust the font size if needed
  escapeinside=||,
  % Any other options you need
}
\usepackage{mdframed}


\usepackage{lipsum}
\usepackage{listings}
% Rename "Listing: " to "Extract: "
\renewcommand{\lstlistingname}{Extract}

\usepackage{hyperref}


% NOTE: make hyperref and algorithmic work together better:
\newcommand{\theHalgorithm}{\arabic{algorithm}}

% \usepackage{icml2025} % NOTE: will display the line numbers
\usepackage[accepted]{./settings/icml2025} % NOTE: won't display the line numbers
% \usepackage[nohyperref, accepted]{icml2025} % NOTE: won't display the line numbers

% For theorems and such
\usepackage{amsmath}
\usepackage{amssymb}
\usepackage{mathtools}
\usepackage{amsthm}

% if you use cleveref..
\usepackage[capitalize,noabbrev]{cleveref}

%%%%%%%%%%%%%%%%%%%%%%%%%%%%%%%%
% THEOREMS
%%%%%%%%%%%%%%%%%%%%%%%%%%%%%%%%
\theoremstyle{plain}
\newtheorem{theorem}{Theorem}[section]
\newtheorem{proposition}[theorem]{Proposition}
\newtheorem{lemma}[theorem]{Lemma}
\newtheorem{corollary}[theorem]{Corollary}
\theoremstyle{definition}
\newtheorem{definition}[theorem]{Definition}
\newtheorem{assumption}[theorem]{Assumption}
\theoremstyle{remark}
\newtheorem{remark}[theorem]{Remark}

% TODO: is useful during development; simply uncomment the next line
%    and comment out the line below the next line to turn off comments
%\usepackage[disable,textsize=tiny]{todonotes}
\usepackage[textsize=tiny]{todonotes}

\usepackage{multirow}
\usepackage{multicol}
\usepackage{enumitem}
% c.f. Efficient Online Reinforcement Learning Fine-Tuning Need Not Retain Offline Data
\usepackage[most,skins,theorems]{tcolorbox}
\tcbset{
  aibox/.style={
    width=\linewidth,
    top=8pt,
    bottom=4pt,
    colback=blue!6!white,
    colframe=black,
    colbacktitle=black,
    enhanced,
    center,
    attach boxed title to top left={yshift=-0.1in,xshift=0.15in},
    boxed title style={boxrule=0pt,colframe=white,},
  }
}

\newtcolorbox{AIbox}[2][]{aibox,title=#2,#1}

% NOTE: apparently when disabled latex will try to stretch the page's content to the bottom in order to fill the whole page
% \raggedbottom

\renewcommand{\ref}[1]{\textbf{\ref{#1}}}

\begin{document}

\twocolumn[
    % \begin{frame}
    %     \centering
    %     \includegraphics[height=0.6125cm]{../../assets/logos/cristal-lab-black.logo.png}
    %     \hspace{1em}
    %     \vrule width 0.5pt
    %     \hspace{1em}
    %     \includegraphics[height=0.6125cm]{../../assets/logos/universite-lille.logo.png}
    %     \hspace{1em}
    %     \vrule width 0.5pt
    %     \hspace{1em}
    %     \includegraphics[height=0.6125cm]{../../assets/logos/universite-rouen.logo.png} 
    % \end{frame}
    
    \vskip 0.75cm

    \icmltitlerunning{Bouldering Video Segmentation}
    \icmltitle{Bouldering Video Segmentation \\ \large\textit{M1 Data Science Research Project} \\ \small\textit{\underline{Centrale Lille} - \underline{Université de Lille} - \underline{IMT Nord Europe} - \underline{Cristal Lab} - \underline{Université de Rouen}}}

    \icmlsetsymbol{equal}{*}

    \begin{icmlauthorlist}
        \icmlauthor{Nadir Kichou}{ul}
        \icmlauthor{Jérémie Boulanger}{sigma}
        \icmlauthor{Ludivine Plumhans}{ur}
        \icmlauthor{Ludovic Seifert}{ur}
    \end{icmlauthorlist}

    \icmlaffiliation{ul}{Université de Lille}
    \icmlaffiliation{sigma}{CRIStAL Lab, SIGMA}
    \icmlaffiliation{ur}{Université de Rouen}

    \icmlcorrespondingauthor{Nadir Kichou}{nadir.kichou.etu@univ-lille.fr}
    \icmlcorrespondingauthor{Jérémie Boulanger}{jeremie.boulanger@univ-lille.fr}
    \icmlcorrespondingauthor{Ludivine Plumhans}{ludivine.plumhans@univ-rouen.fr}
    \icmlcorrespondingauthor{Ludovic Seifert}{ludovic.seifert@univ-rouen.fr}
    

    \icmlkeywords{Temporal Action Segmnetation, TAS, Video Machine Learning, Machine Learning, ICML}

    \vskip 0.75cm

    \newpage
]

\printAffiliationsAndNotice{}

\begin{abstract}
    This research project focuses on temporal action segmentation in professional bouldering videos.
    The objective is to classify each frame of a video based on the climber’s activity to enable a detailed analysis of their performance.
    The project leverages pre-trained networks, which are crucial for addressing the dataset's limited size by providing robust spatio-temporal feature extraction.
    These features are then combined with either a Multi-Layer Perceptron (MLP) or a Long Short-Term Memory (LSTM) network for classification.
    The dataset, collected during a bouldering competition, presents several challenges, including class imbalance, noise, and limited size.
    \\\\
    Experimental results demonstrate that models incorporating temporal context significantly outperform frame-based approaches.
    Notably, the X3D-M model with an LSTM classifier achieves the highest accuracy of 86.61\%.
    This work contributes valuable insights into effective model selection for bouldering video analysis and highlights potential improvements for real-world applications.
    % Future work will focus on enhancing real-time capabilities, exploring alternative loss functions, and expanding the dataset through semi-supervised learning techniques.
    The code and dataset used in this project are available at: \url{https://github.com/raideno/bouldering-video-segmentation}.
\end{abstract}

% \section{Introduction}
\lipsum[1-2]

\missingfigure{Make sure to include a figure here.}
\section{Context}
\label{section:context}

\noindent\textbf{Bouldering Basics.} 
It is a form of rock climbing where climbers tackle short but challenging routes, typically lasting around 4 minutes. During this time, climbers have the freedom to choose their climbing strategies, retry as many times as needed, and aim to complete the route as quickly as possible. The ultimate goal is to reach the top of the climb in the least amount of time.

\noindent\textbf{Project Overview.} 
This research project is part of a larger initiative (ANR) aimed at improving bouldering performance. The focus is on developing tools for coaches to analyze bouldering performances, allowing them to provide better advice to help climbers improve.

\noindent\textbf{Existing Tools.} 
Among these tools are: Grip Detection, Path Tracking, Performance Analysis, and more. Several models have already been developed as part of this broader effort, including a model for detecting the grips used by the climber and another for tracking the climber's path during a climb.

\noindent\textbf{Phase Identification.} 
These models provide valuable data, but in order to fully analyze a climber's performance, it is essential to distinguish between the different phases of a bouldering event (Climbing, Observing, Brushing). By doing so, we can apply the appropriate model to each phase and gather the corresponding statistics.

\noindent\textbf{Research Focus.} 
This brings us to my current research project. I will be focusing on Temporal Action Segmentation (TAS), a technique that will allow us to run the grip and path analysis models at the correct timestamps. This step is crucial for gaining deeper insights into the different phases of climbing, such as observing, reading the route, and brushing the grips. By analyzing the durations, order, and impact of these phases on a climber's final performance, we can provide more precise guidance to coaches and improve training methods.

\begin{figure*}[t]
    \centering
    \begin{tabular}{@{}c@{\hspace{15pt}}c@{\hspace{15pt}}c@{\hspace{15pt}}c@{}}
      \setlength{\fboxsep}{0pt}
      \fbox{\includegraphics[width=0.2\textwidth]{../../assets/images/observing.2.png}} &
      \setlength{\fboxsep}{0pt}
      \fbox{\includegraphics[width=0.2\textwidth]{../../assets/images/brushing.3.png}} &
      \setlength{\fboxsep}{0pt}
      \fbox{\includegraphics[width=0.2\textwidth]{../../assets/images/climbing.1.png}} &
      \setlength{\fboxsep}{0pt}
      \fbox{\includegraphics[width=0.2\textwidth]{../../assets/images/climbing.2.png}} \\[6pt]
      (a) Observing the block. &
      (b) Brushing the grips. &
      (c) Climbing the block. &
      (d) Climbing the block.
    \end{tabular}
    \caption{Different phases of bouldering.}
    \label{fig:phases-of-bouldering}
  \end{figure*}
\section{Problem Setup}

\begin{tcolorbox}[colback=lightgray!10, colframe=black, title={Research Aim}]
    This project focuses on the task of temporal action segmentation in professional bouldering videos. The goal is to classify each frame based on the climber's activity, enabling a more detailed analysis of the climber's performance throughout the video.
\end{tcolorbox}

\subsection{Video Representation}
We represent each video as a sequence of frames $video_i = \{f_1, f_2, \dots, f_{N_i}\}$, where each frame $f_i$ corresponds to the image captured at position $i$ in the video. Each frame $f_i$ is a tensor with shape $(3 \times H \times W)$, where $3$ denotes the number of channels, $H$ is the height of the video, and $W$ is the width of the video. The total number of frames in the video is denoted by $N_i$.

The $i$-th video's duration, $d_i$, in seconds is given by $d_i = N_i / F$, where $F$ is the frame rate of the video. In our experiments, we fix $F = 25 \, \text{fps}$, which matches the frame rate used in our dataset.

\subsection{Annotation Representation}  
Each video's annotations are provided at the frame level. Given a video $video_i$, its corresponding annotations are defined as $a_i = \{c_1, c_2, \cdots, c_{N_i}\}$, where $c_j \in C$ is the annotation corresponding to the frame $f_i$, and $C = \{\text{Climbing, Brushing, Observing, Stopwatch}\}$ is the set of all possible activity labels.

Alternatively, the annotations can be represented using start and end timestamps. In this format, the annotations are defined as a set of tuples:  
\[
\{(s_1, e_1, c_1), (s_2, e_2, c_2), \dots \}
\]  
where $s_j$ is the index of the starting frame for the $j$-th action segment, $e_j$ is the index of the ending frame, and $c_j \in C$ is the corresponding action label.

Depending on the context, the starting frame $s_j$ and ending frame $e_j$ can alternatively be expressed as timestamps (e.g., seconds, milliseconds) or as a starting timestamp combined with a duration. 

This alternative representation is more compact and especially useful when dealing with longer videos or when visualizing action intervals and it is the preferred format when annotating videos as it is easier to deal with. While the first representation provides more precision at the frame level, the second representation offers a higher-level view of the video’s structure, facilitating analysis of action segments and their durations.

Note that it is easy to convert between these two representations by computing the starting and ending frames from the timestamps and vice versa.

\subsection{Problem Formulation}
The goal is to develop a model $\mathcal{M}$ that takes a sequence of frames as input and predicts the corresponding sequence of frame-level annotations, $\mathcal{M}: \{f_1, f_2, \dots, f_{N_i}\} \rightarrow \{c_1, c_2, \dots, c_{N_i}\}$.

\noindent\textbf{Frame-wise Prediction.}  
In this formulation, the model processes each frame independently and predicts its corresponding label. As the each frame is passed through a convolutional layer or a similar architecture. The model predicts a label $c_j$ for each frame $f_j$, i.e.,

\[
c_j = \mathcal{M}(f_j)
\]

While this approach is computationally efficient, it may struggle to capture temporal dependencies between consecutive frames.

\noindent\textbf{Sequence-based Prediction.}  
Here, the model processes a sequence of $T$ consecutive frames and predicts a single label for the entire sequence. A larger segment length $T$ allows the model to leverage more temporal context, improving its ability to recognize complex patterns. This can be expressed as:

\[
c_{j:j+T-1} = \mathcal{M}(f_j, f_{j+1}, \dots, f_{j+T-1})
\]

In this approach, the model predicts the activity label for the entire sequence of frames, taking into account their temporal dependencies. However, this comes at the cost of increased computational complexity.

The choice between these approaches depends on the desired trade-off: a smaller segment length favors computational efficiency, while a larger segment length enhances the model's capacity to capture temporal dependencies. In the following $T$ is taken between $4$ and $32$ frames depending on the model's variant.

\subsection{Additional Statistical Measures}

Besides the usual statistics, such as the mean and average duration, we also utilize the following two metrics, introduced in \cite{tas-survey}, to gain further insights into the data:

\noindent\textbf{Repetition Score.}  
It quantifies the degree of repetition of actions within a video. Using the notations defined earlier, it is expressed as:  

\[
\text{Repetition Score} = \frac{|\{c_j \mid c_j \in a_i\}|}{|a_i|}
\]

where \( |\{c_j \mid c_j \in a_i\}| \) denotes the number of unique action labels in the annotation sequence \( a_i \), and \( |a_i| \) represents the total number of annotated actions in the video. This score ranges from 0 to 1, where 0 indicates no repetition of actions, and 1 indicates that the same action is repeated throughout the video.

\noindent\textbf{Order Variation Score.}  
It measures the consistency of the order of actions across different videos or sequences. Using the previously defined notations, it is expressed as:  

\[
\text{Order Variation Score} = \frac{1}{|V|} \sum_{i, j \in V} d(a_i, a_j)
\]

where \(d(a_i, a_j)\) represents the pairwise distance between the annotation sequences \(a_i\) and \(a_j\), and \(V\) is the set of all videos in the dataset. The score ranges from 0 to 1, where 0 indicates identical action order across all videos, and 1 indicates completely inconsistent action order.  

\subsection{Evaluation Metrics}

During the project we are going to use the \textbf{Accuracy} as a metric to evaluate the model's performance. It is defined as:

$$
\text{Accuracy} = \frac{\text{Correct Predictions}}{\text{Total Predictions}}.
$$
\noindent\textbf{\small{Pros.}} Simple to compute and provides a clear overall performance indicator. \textbf{\small{Cons.}} Not informative for imbalanced classes, as high accuracy can be achieved by predicting the majority class.

\begin{AIbox}{Pyhton Package - TAS Helpers.}
    We introduce the \texttt{tas\_helpers} (Temporal Action Segmentation Helpers) library, which contains implementations of various metrics and scores discussed above. Additionally, it offers utilities for visualizing video segmentations and more. The package is available at: \url{https://github.com/raideno/tas-helpers}.
\end{AIbox}

\section{Dataset}

Here we'll present our dataset just like in the survey paper and we'll present a bunch of statistics about it.

\subsection{Raw Format}
We talk about how the dataset was constructed, that it was constructured by non data scientists, that it was in exceel files and in a format not appropriate for training.

Annotations are shifted, videos don't start immediately on the climbing but often begin with some sort of synchronization sequence.

\subsection{Chosen Structure}
Here we state the different possible structures \& present the choosen one.

\subsection{Developed Tools \& Easier Training}

\href{https://github.com/raideno/cached-dataset}{https://github.com/raideno/cached-dataset}
\href{https://github.com/raideno/video-dataset}{https://github.com/raideno/video-dataset}

\subsection{Statistics, Illustrations \& Data Exploration}

\begin{figure}[ht]
    \centering
    \includegraphics[width=8.4cm]{assets/mwe/example-image-a}
    \caption{Your caption here.}
    \label{fig:example}
\end{figure}

Showcase some example image, sequence of frames \& videos of the dataset.

Display some plots that showcase the dataset statistics and analyze and talk about these statistics.

\subsection{Optimization}

\todo[inline]{This section was intended for talking about the cached-dataset and how we used it to avoid re-extracting the features each time, but we'll probabily discard it and move it lower as we haven't introduced the model yet to talk about caching and optimizing it's training.}

\subsection{Other Datasets}

Talk about other popular datasets in the field, their sizes and magnitude, where they are speciliazed, by who and why they have been created and the kind of videos we can find in them.
\section{Related Work}

This will be a sort of state of the art.

We do an overview of TAS methods and other formulations methods.

We talk about the limitations in the current methods (not enough data and compute power available, not enough diverse data, as seen most of the datasets are youtube videos).

Illustration that describe the evolution of the different methods with the evolution of dataset size and compute power.
Hand Features - Optical Flow - 2D CNNs - 3D CNNs
Talk about the different approaches and techniques. Talk about their evolution just like it is done in \textbf{ViViT: A Video Vision Transformer} paper. (First started by using hand collected features, 2D CNNs, 3D with optical flow, full 3d, transformers as the datasets started to grow, etc.)

\todo[inline]{We'll need an illustration here.}

\todo[inline]{We could also have a cahrt showing the evolution of accuracy and number of papers on the field with time, and on the time axis we put markers for different events such as the release of popular datasets, improvements in compute power, etc.}

\todo[inline]{We also need to talk about the old school methods, how they used optical flow, etc}

\todo[inline]{Maybe this should be moved elswhere, i feel it is not it's place and should be presented before, maybe in the formulations section}
Talk about the importance of TAS and its applications.

Talk about the limitations of this presented methods, their constraints in terms of dataset size and required compute power.

NOTE: the models tend to drift much faster when it comes to videos data as the distributions are much much larger and thus it requires even more data  probabily.

NOTE: recall that it is a hard problem and that video data is very very varied and diverse and that distributions aren't the same and that even pre trained models don't work very well on other data then thaier datasets, and it is an on going research area.
\section{Our Approach}

\begin{figure*}[t]
    \centering
    \includegraphics[width=\textwidth]{../../assets/figures/model-overview.png}
    \caption{Model Architecture Overview.}
    \label{fig:your-label}
\end{figure*}

Given the dataset size constraint, training a neural network from scratch with our data is infeasible and unlikely to yield meaningful results. To address this, we leverage pre-trained networks trained on large-scale datasets that contain similar types of actions to those found in our data. These actions are broad, non-atomic, and not highly granular, making them comparable to activities present in popular datasets like Kinetics.

Our strategy involves utilizing these pre-trained networks to extract feature representations from our videos, clips, and frames. Specifically, we extract features from the hidden layers of these networks — typically the layer immediately preceding the final classification layer — as they encode rich spatio-temporal information relevant to our task. These extracted features are then fed into a custom network designed for classification on our custom action labels.

This approach addresses multiple constraints simultaneously. First, it mitigates the impact of our limited dataset by capitalizing on robust features learned from extensive pre-training. Second, it reduces the computational burden, as we avoid the costly process of training a deep network from scratch. Finally, this method is well-suited to our project timeline, as designing and training a complex architecture would exceed our allocated 120 hours.

In the following sections, we describe the key components of our approach and how we train it.

\subsection{Spatio-Temporal Feature Extractors}
The backbone of our model is a spatio-temporal feature extractor designed to capture both spatial and temporal information from video data. We employ pre-trained models for this component, freezing their weights to retain the learned representations.

Two types of models are explored for this purpose: frame-level feature extractors and segment-level feature extractors.

\subsubsection{Frame-Level Extractors}
Frame-level extractors process individual frames independently to extract spatial features. Examples include 2D CNNs such as ResNet, vision transformers like DINO, CLIP, or ViT, and potentially handcrafted features such as object presence (e.g., brushes) or positional key points.

While handcrafted features can be informative, they introduce biases and require additional training for object detection, which is resource-intensive. Thus, our approach prioritizes automated feature extractors to maximize efficiency and robustness.

Once frame-level features are extracted, we explore several strategies to aggregate these features into a single representation for each video segment:

\noindent\textbf{\small{Averaging.}} The most common method, recommended in the literature for its simplicity and effectiveness.  
\noindent\textbf{\small{Addition.}} Similar to averaging but involves summing the features directly.  
\noindent\textbf{\small{Concatenation.}} Combines frame features without additional operations.  
\noindent\textbf{\small{Temporal Modeling.}} Uses models like LSTMs or Transformers to capture temporal dependencies, offering improved contextual understanding at the cost of higher computational complexity.

\subsubsection{Segment-Level Extractors}
Segment-level extractors process multiple consecutive frames directly, capturing both spatial and temporal patterns. Examples include 3D CNNs, which extend 2D CNN architectures to model motion and temporal dynamics, and transformer-based models that have demonstrated strong performance on video tasks. While effective, these models are typically more computationally demanding.

\subsection{Temporal Sub-Sampling}
To improve efficiency without sacrificing performance, we incorporate temporal sub-sampling. This technique involves reducing the frame rate before feature extraction, as consecutive frames in bouldering videos often exhibit minimal visual change. Studies in the field highlight that temporal sub-sampling is a simple yet highly effective method for reducing computational load without significant performance degradation.

\subsection{Temporal Context Window}
The temporal context window defines the number of consecutive frames the model processes at once. Choosing an optimal window size is crucial:

\noindent\textbf{\small{Short Window.}}
It may not capture sufficient context for accurate classification.

\noindent\textbf{\small{Long Window.}}
It risks embedding the entire video into a single feature, increasing computational costs and reducing model flexibility.

As the temporal context window of the backbone networks is quite limited to around 16-32 frames usually, processing the whole 4 minutes of video and extracting a single feature for the entire video is both computationally costly and not practical in our case. This would be more practical for video classification tasks, where the goal is to classify the whole video. In our case, we aim to extract different time segments and classify them individually. This is why we present two approaches for processing these time segments.

Balancing these factors is key to achieving optimal performance.

By combining pre-trained feature extractors, temporal sub-sampling, and thoughtful hyperparameter tuning, our approach aims to achieve robust performance despite dataset limitations.

\subsection{Classifier}

This is the component of our network that'll receive the spatio temporal features corresponding to a video segment / clip and output the class of the action that is being performed in this segment.

For this we consider two approaches, the first one is to use a simple MLP with a single hidden layer and the second one is to use an LSTM with a single layer. The idea behind the use of the LSTM is that it can capture the temporal dependencies between the frames / segments and thus potentially improve the model's performance.

We could also use an attention mechanism to allow the model to focus on the most important frames / segments but this would increase the model's complexity and thus the training time and the computational cost.

Another possible improvement would be to use fixed or learnable positional embeddings in combination with the MLP in order to incorporate temporal awareness into the MLP. 

%  --- --- ---

\subsection{Training Methodology}

\subsubsection*{Backbone Model}
We experimented with several backbone models for feature extraction, leveraging both frame-level and segment-level networks. Notable examples include:

\noindent\textbf{\small{DINO}} and \noindent\textbf{\small{CLIP.}} Vision Transformers used as frame-level feature extractors for their strong spatial representation capabilities.

\noindent\textbf{\small{X3D}}, \noindent\textbf{\small{R3D}}, and \noindent\textbf{\small{Slowfast.}} Segment-level extractors known for their robust spatio-temporal modeling.

Most of the backbones we use are pre-trained or fine-tuned on the Kinetics dataset, except for CLIP, which leverages large-scale internet data.

\subsubsection*{Loss Function}
Our classification model is trained using the \textbf{cross-entropy loss} function, defined as:

\[
\mathcal{L} = -\sum_{i=1}^{N} y_i \log \hat{y}_i
\]

where \(y_i\) is the true label and \(\hat{y}_i\) is the predicted probability for class \(i\). While cross-entropy loss effectively handles multi-class classification, alternative losses such as focal loss or label-smoothing loss have been proposed in the literature to address class imbalance and calibration issues. These approaches may be worth exploring in future work.

\subsubsection*{Classifier Model}
\todo[inline]{Describe the architecture of the MLP and LSTM model that are going to be used for classification.}

\subsubsection*{Training Setup}
The training pipeline was configured as follows:

\noindent\textbf{Optimizer.} Adam optimizer with a learning rate of $1 \times 10^{-4}$.

\noindent\textbf{Batch Size.} 32 samples per batch for efficient convergence.

\noindent\textbf{Early Stopping.} Applied based on validation loss with a patience of 5 epochs to prevent overfitting.

\noindent\textbf{Batch Structure.} Each batch consists of video segments rather than entire videos. This approach ensures manageable memory usage and allows the model to learn from diverse temporal contexts within each batch.

To prevent data leakage during validation, we ensured that segments from the same video were never split across training and validation sets. This avoids the model unintentionally learning video-specific patterns rather than generalizable features.

\subsubsection*{Data Filtering and Preprocessing}
To improve data quality and enhance model performance, we applied several preprocessing steps:

\noindent\textbf{Removal of Personless Frames.} Frames without visible climbers were excluded to reduce noise.

\noindent\textbf{Cleaning of Low-Quality Data.} Unannotated frames were removed rather than assigning them a generic "nothing" label. This improved class balance and reduced label ambiguity.

\noindent\textbf{Stopwatch Class Removal.} We excluded segments showing only the competition timer, as these frames added no meaningful information to the model's predictions.

\begin{AIbox}{Python Package - Cached Dataset.}
    To accelerate training, we developed a custom Python package called \textbf{cached-dataset} - \href{https://github.com/raideno/video-dataset}{https://github.com/raideno/video-dataset}. This tool caches the transformation of a dataset to disk, significantly reducing data loading overhead during training and feature extraction. This allow us to efficiently iterate through training experiments and hyperparameter tuning.
\end{AIbox}

By combining robust pre-trained feature extractors, temporal sub-sampling, and thoughtful data filtering strategies, our training methodology effectively addresses the challenges posed by our limited dataset. These strategies ensured a stable and efficient training process while maximizing model performance.
\section{Results}

\begin{table}
    \centering
    \small
    
    \resizebox{1\linewidth}{!}{
    \begin{tabular}{lcllr}
    \toprule
    Backbone & \multicolumn{1}{c}{MLP-Acc} & \multicolumn{1}{c}{MLP-Acc} & \multicolumn{1}{c}{LSTM-Acc} & \#Parameters \\
    % Backbone & Type & MLP-Acc & LSTM-Acc & \#Parameters \\
    \midrule
    yolo & Frame & 65.01\% ± 4.41 & 69.94\% ± 2.40  & 2.9M \\
    dino & Frame & 80.58\% ± 4.56  & 83.20\% ± 3.26 & 22.1M \\
    r3d & Segment & 84.27\% ± 5.03  & 85.68\% ± 3.64 & 31.6M \\
    i3d & Segment & 76.53\% ± 8.81  & 79.38\% ± 5.11 & 12.7M \\
    clip & Frame & 76.49\% ± 2.36   & 79.92\% ± 2.74  & 151.3M \\
    x3d-xs & Segment & 82.11\% ± 3.73 & 83.87\% ± 2.68  & 3.0M \\
    x3d-s & Segment & \textbf{85.28\% ± 4.54} & 85.84\% ± 3.63  & 3.0M \\
    x3d-m & Segment & 85.01\% ± 4.75 & \textbf{86.61\% ± 2.32}  & 3.0M \\
    x3d-l & Segment & 84.65\% ± 4.10 & 85.91\% ± 2.32  & 5.3M \\
    s3d-k & Segment & 78.08\% ± 5.20 & 78.04\% ± 3.33  & 7.9M \\
    s3d-h & Segment & 59.37\% ± 9.10 & 47.19\% ± 6.22  & 9.7M \\
    slowfast & Segment & 84.22\% ± 3.07  & 82.49\% ± 4.61 & 33.6M \\
    vivit & Segment & 78.42\% ± 3.70  & 81.41\% ± 3.75 & 88.6M \\
    \bottomrule
    \end{tabular}
    }
    \vspace{-2ex}\caption{MLP Models Performance Results.}
    \end{table}
    

\begin{figure*}[!htb]
    \centering
    \begin{minipage}{0.48\textwidth}
        \centering
        \includegraphics[width=1\textwidth]{../../assets/figures/mlp.training-results.boxplot.png}
        \caption{MLP Model Training Results}
        \label{fig:mlp-training-results}
    \end{minipage}
    \hfill
    \begin{minipage}{0.48\textwidth}
        \centering
        \includegraphics[width=1\textwidth]{../../assets/figures/lstm.training-results.boxplot.png}
        \caption{LSTM Model Training Results}
        \label{fig:lstm-training-results}
    \end{minipage}
\end{figure*}

\subsection{Performance Analysis}
Table~\ref{table:training-results} presents the classification accuracy of various backbone architectures combined with either MLP or LSTM classifiers for bouldering video segmentation. The results reveal several interesting patterns that provide insights into the effectiveness of different approaches for this specific task.

\subsubsection{Segment vs. Frame-based Extraction}
Our experiments demonstrate a clear advantage for models that incorporate temporal information through segment-level processing. As shown in Table~\ref{table:training-results}, segment-based models consistently outperform frame-based models, with the top-performing models (\modelname{X3D family}) all utilizing temporal information. The \modelname{X3D-S} model achieves 85.28\% accuracy with an MLP classifier, while \modelname{R3D} reaches 86.61\% accuracy when combined with an LSTM. This pattern aligns with the intuitive understanding that climbing actions involve temporal dynamics that cannot be fully captured by analyzing individual frames or segments in isolation.

\subsubsection{Analysis of Underperforming Models}
Two backbone architectures exhibit notably lower performance compared to others:

\noindent\textbf{YOLO-based Skeleton Features.}
The \modelname{YOLO}-based approach, which extracts skeleton key points, achieves only 65.84\% accuracy with MLP and 70.61\% with LSTM. This underperformance can be attributed to the similarity in climber movement dynamics across different action categories. For instance, the speed and pattern of movement during observing, brushing, and climbing activities may exhibit similar skeleton motion signatures despite being semantically distinct. A potential improvement would be to incorporate absolute spatial positions rather than positions relative to the climber's center of mass. However, this approach would introduce camera position dependency, potentially reducing generalizability across different recording setups.

\noindent\textbf{S3D with HowTo100M Pre-training.}
The \modelname{S3D} model pre-trained on HowTo100M (\modelname{S3D-H}) shows particularly poor performance (59.52\% with MLP, 43.25\% with LSTM). This can be explained by the nature of the pre-training dataset, which consists primarily of instructional videos featuring fine-grained hand manipulations and subtle movements. The same architecture pre-trained on the Kinetics dataset (\modelname{S3D-K}), which contains more diverse and dynamic whole-body activities, performs substantially better (78.38\% with MLP, 78.57\% with LSTM). This significant performance gap highlights the critical importance of selecting appropriate pre-training datasets that align with the target domain's action characteristics.

\subsubsection{Model Complexity and Performance}
Interestingly, our experiments reveal that larger models do not necessarily yield better performance for this task. The \modelname{X3D-S} model (3.0M parameters) outperforms the larger \modelname{X3D-L} variant (5.3M parameters) when using an MLP classifier. Similarly, the relatively lightweight \modelname{R3D} backbone (31.6M parameters) achieves better results to much larger models such as \modelname{CLIP} (151.3M parameters) and \modelname{ViViT} (88.6M parameters). This suggests that for our specific dataset size and task complexity, model architecture design is more important than raw parameter count. The \modelname{X3D} family, designed specifically for efficient video understanding, demonstrates excellent performance-to-parameter ratios across all variants.

\subsubsection{Temporal Modeling with LSTM}
When comparing MLP and LSTM classifiers, we observe that LSTM generally provides modest improvements across most backbones. For instance, \modelname{DINO} shows a 2.11\% improvement when switching from MLP to LSTM. However, this pattern is not universal—SlowFast actually performs 1.73\% worse with LSTM compared to MLP. The inconsistent benefits of additional temporal modeling through LSTM may be due to our dataset size constraints, as larger datasets typically show more significant improvements from temporal modeling, as demonstrated in prior work \cite{action-clip}.

Beyond improvements in accuracy, we also observe a reduction in variance across multiple runs when using LSTM classifiers. For example, the \modelname{X3D-M} model shows a standard deviation of 4.16\% with MLP but only 2.22\% with LSTM, representing a nearly 50\% reduction in variance. This increased stability is a significant advantage in practical applications, as it indicates more reliable and consistent performance across different climbing sessions and environmental conditions.

\subsection{Practical Implications}
Based on our experimental results, we can draw several conclusions to guide model selection for practical bouldering video segmentation applications:

\noindent\textbf{For accuracy-critical applications.}
The \modelname{R3D} with LSTM classifier provides the highest overall accuracy (86.80\%) and represents the best choice when classification performance is the primary concern.
    
\noindent\textbf{For resource-constrained environments.}
The \modelname{X3D-XS} with MLP classifier offers an excellent balance between accuracy (81.64\%) and computational efficiency, utilizing only 3.0M parameters. And this model provide the best trade-off between speed and accuracy.
    
These findings provide valuable insights for climbing gym operators, sports coaches, and performance analysts working with climbing videos. For coaches analyzing technique, the highest accuracy models would better distinguish between different climbing phases, while training facilities with limited computing resources could implement the lighter models for real-time feedback systems. The ability to reliably segment climbing activities enables more targeted training programs and better performance assessment for climbers at all levels.
\section{Limitations \& Future Directions}

While our approach demonstrates promising results for bouldering video segmentation, several limitations remain to be addressed in future work.

\subsection*{Methodological Limitations}
Our current implementation faces several methodological constraints. First, the cross-entropy loss function used may not be optimal for temporal segmentation tasks with class imbalance. Alternative losses could potentially improve performance on underrepresented classes. Second, despite the strong performance of convolution architectures, we did not extensively explore transformer-based approaches specifically designed for video understanding, such as TimeSformer or MViT, which have shown state-of-the-art results in related action recognition tasks.

Real-time processing remains a significant limitation of our current approach. The segment-based models that achieved the highest accuracy operate with inherent latency due to their temporal window requirements. For practical applications coaches usually don't require realtime annotations.

\subsection*{Data-Related Challenges}
Limited training data remains a fundamental constraint. Our dataset, while sufficient to demonstrate the viability of automated bouldering segmentation, lacks the scale and diversity needed for robust real-world deployment. We identify several potential approaches to address this:

\noindent\textbf{Semi-supervised Learning.}
Using our best-performing models to pseudo-label unlabeled climbing videos, then leveraging these annotations for additional training data. This bootstrapping approach could substantially increase our effective dataset size while maintaining a manually verified validation set to avoid bias.

\noindent\textbf{External Data Integration.}
Augmenting training with targeted external data from sources like YouTube or subsets of action recognition datasets like Kinetics that contain climbing or similar activities. This would require careful selection and possibly domain adaptation techniques to ensure relevance.

\noindent\textbf{Data Augmentation.}
While we employed basic augmentation techniques, more advanced video-specific augmentations such as temporal shifts, speed variation, and view synthesis could improve model robustness to variations in climbing styles and camera positions.
\section{Conclusion}

\subsection*{Research Contributions}
This work demonstrates the feasibility of automated bouldering video segmentation using modern deep learning approaches. Our experiments revealed that segment-based models, particularly the X3D family and the R3D, outperform frame-based approaches for this task, achieving accuracies of up to 86.80\%. We established that the choice of pre-training dataset significantly impacts performance, as evidenced by the stark difference between S3D models trained on HowTo100M versus Kinetics. Notably, model complexity did not necessarily correlate with performance, with lighter models often matching or exceeding their larger counterparts.

Future research should focus on improving real-time capabilities, investigating alternative loss functions, and exploring transformer-based architectures for video understanding. Semi-supervised approaches leveraging unlabeled climbing footage could address the data scarcity challenge while maintaining evaluation integrity.

\subsection*{Project Reflection}
This project provided valuable experience in end-to-end data science development. Building a computer vision system from scratch presented numerous challenges, from data collection and annotation to model selection and evaluation. The need to balance performance requirements against computational constraints mirrors real-world AI deployment scenarios.

The most significant challenge was navigating the vast landscape of video understanding models without extensive training resources. This required careful consideration of transfer learning approaches and creative solutions for data augmentation.

This work served as a formative experience in independent research, requiring comprehensive exploration of the computer vision literature and adaptation of techniques across domains. It demanded proficiency across the entire machine learning pipeline—from data preparation to deployment considerations—providing practical skills beyond what typical structured projects offer. This holistic approach to solving a complex visual understanding problem has reinforced both technical capabilities and research methodology fundamentals.

\section*{Acknowledgment}
I would like to express my sincere gratitude to my supervisor, Jeremie Boulanger, for his invaluable guidance, support, and encouragement throughout this project.

\newpage

\section*{Bibliography}

\bibliography{reference}
\bibliographystyle{icml2025}


\newpage
\appendix
\onecolumn

% The $\mathtt{\backslash onecolumn}$ command above can be kept in place if you prefer a one-column appendix, or can be removed if you prefer a two-column appendix.  Apart from this possible change, the style (font size, spacing, margins, page numbering, etc.) should be kept the same as the main body.

\section{Code}

\todo[inline]{Describe the different small functions used for computing the different scores and metrics.}

\section{Related Work}

\todo[inline]{Describe in more details each backbone network.}

\newpage
\section{Figures and Tables}

\end{document}

